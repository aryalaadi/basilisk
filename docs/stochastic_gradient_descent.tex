\documentclass[a4paper,11pt]{article}
\usepackage{amsmath}
\title{\vspace{-1.0cm}basilisk: on implementing SGD optimization}
\author{Aryal Aaditya}
\begin{document}
  \pagenumbering{arabic}
  \maketitle
  \pagenumbering{arabic}
  \section{abstract}
	The stochasitc gradient descent(SGD) is a basic machine 
	learning optimization algorithm which updates a layer's weights to approach of local minima. This documentation 
  details how the basilisk framework implements stochastic gradient descent.

  \section{theory}
  The stochastic gradient descent minimizes a cost function.
  The cost function returns an estimate of how accurately can the current weights and biases compute a given pattern.
  The stochastic gradient descent optimizer then approaches a local minima every iteration, and adjusts the weights until
  the model can accurately compute a given pattern. The optimization itself can be expressed with a single equation:
  \begin{equation}
    w := w - \eta \nabla Q(w) 
  \end{equation}
  where, \(w\) is the weight to be adjusted, \(Q(w)\) being the 
  cost function used to estimate the accuracy and \(\eta\) being the learning rate.
  
  \section{implementation}
  The following code snippet is the basic implementation of the gradient descent. 
  The optimizer function, ironically is unoptimzied. 
  This method is an implementation of the BASOptimizer struct. The \(h\) parameter is a matrix of infinitesimals.
  x1 variable computes \(\nabla Q(w)\) which also can be expressed as
  \(\lim_{x \to \infty}\dfrac{Q(w+h)-Q(w)}{h}\), a simple partial finite difference.
  After the finite difference has been computed, the weight is adjusted by subtracting the finite difference from the previous weight. The weights when iterated and trained on multiple datasets approach a certain point where it can accurately compute a given pattern with a certain accuracy. 
  The \(d\) parameter is one unit of the dataset, which is used to compute the 
  cost function. This optimization is done with every single unit of data set provided to train, for multiple epohs until a local minima has been approached.
  \begin{verbatim}
src/basoptimizer.rs:29

pub fn optimize(self, mut model: BASModelSEQ, d: &[BASMatrix; 2], layer: usize, rate: f64) {
  let mut wx = model.clone();
  wx.layers[layer].weights.scalaradd(self.h);

  let mut x1 = BASCost::loss(BASCost::MSE, &wx, &d);
  let _ = x1.sub(&BASCost::loss(BASCost::MSE, &model, &d));

  x1.scalarmul(rate/self.h);

  let _ = model.layers[layer].weights.sub(&x1);
}
    \end{verbatim}
  
    \section{testing and observation}
    To test the accuracy of the implementation, we will check to see if it can 
    optimize for a simple pattern \(f(x) = 2x\). \newline
    here,

    \(d = \begin{matrix} 
      [2], [4]
    \end{matrix}\) and the model contains a randomized weight.
    This program uses the SGD optimization to optimize the weight with the given dataset \(d\).
    \begin{verbatim}
    tests/seq_sgd_mse.rs: 23
    for _i in 0..1000 {
      optimzer.to_owned().optimize(&mut model, 
        &d, 
        1, 
        0.005);
    }
    \end{verbatim}
    As expected, the randomized weight approaches 2.
\end{document}

